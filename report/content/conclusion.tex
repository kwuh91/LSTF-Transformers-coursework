% add centering to sections
\titleformat{\section}
  {\centering\normalfont\large\bfseries}{\thesection}{1em}{}

\section*{ЗАКЛЮЧЕНИЕ}

% add to toc
\phantomsection
\addcontentsline{toc}{section}{ЗАКЛЮЧЕНИЕ}

% ------------------------------------CONTENT--------------------------------------

В ходе данной курсовой работы была исследована проблема долгосрочного 
прогнозирования временных рядов (LSTF) с использованием моделей на основе 
архитектуры Трансформер. Было установлено, что, несмотря на свою эффективность, 
базовые модели сталкиваются с ограничениями вычислительной сложности и не всегда 
явно моделируют специфические компоненты временного ряда, такие как тренд и 
сезонность.

Основной целью работы являлась разработка гибридной модели, которая бы 
интегрировала сильные стороны нескольких современных подходов для повышения 
точности и эффективности LSTF. Предложенная модель, названная Convformer, 
объединила в себе три ключевые модификации на основе архитектуры Informer:

\begin{itemize}
  \item Сверточный блок ConvStem для более эффективного извлечения локальных паттернов на начальном этапе обработки последовательности.
  \item Механизм внимания FAVOR+ из модели Performer для обеспечения линейной вычислительной сложности при сохранении способности улавливать глобальные зависимости.
  \item Модуль декомпозиции ряда из модели Autoformer для явного разделения и раздельной обработки трендовой и сезонной компонент.
\end{itemize}

В рамках практической части было проведено экспериментальное сравнение 
предложенной модели Convformer с оригинальной моделью Informer и другими 
базовыми архитектурами на публичном датасете ETTh1. Результаты продемонстрировали, 
что гибридная модель показывает статистически значимое улучшение качества 
прогнозов по метрикам MSE и MAE на большинстве горизонтов прогнозирования. 
Детальный анализ вклада каждого компонента (ablation study) подтвердил, что 
все предложенные модификации вносят положительный вклад, при этом наибольший 
прирост производительности был достигнут за счет внедрения модуля декомпозиции ряда.

Таким образом, данное исследование подтверждает гипотезу о том, что 
синтез специализированных архитектурных решений является перспективным 
направлением для развития моделей прогнозирования временных рядов. 
Объединение механизмов для локальной и глобальной обработки, а также 
явное моделирование структурных компонент ряда позволяет создавать более 
робастные и точные алгоритмы для решения сложных прикладных задач.

% ------------------------------------CONTENT--------------------------------------
