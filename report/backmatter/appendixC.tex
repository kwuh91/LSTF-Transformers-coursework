\phantomsection
\section*{ПРИЛОЖЕНИЕ В}

% add introduction to toc
\addcontentsline{toc}{section}{ПРИЛОЖЕНИЕ В}

% -----------------------------------------------CONTENT-------------------------------------------------

Программный код, используемый в данной работе.\\

% % ------------------------------------------------------------------------------------------------------
% % ----------------------------------------------ConvStem------------------------------------------------
% % ------------------------------------------------------------------------------------------------------

{\noindent\hspace{-12.5pt}\normalsize\bfseries ConvStem}\vspace{-10pt}
\begin{center}
  \begin{lstlisting}[language=Python, 
  caption={Реализация блока ConvStem, состоящего из двух последовательных сверточных слоев с нормализацией и функцией активации.},  
  label={lst:convstem}]
class ConvStem(nn.Module):
    def __init__(self, c_in, d_model):
        super(ConvStem, self).__init__()

        self.conv0 = nn.Conv1d(c_in, d_model, kernel_size=1)

        self.conv1 = nn.Conv1d(in_channels=c_in,
                               out_channels=d_model,
                               kernel_size=5,
                               padding=2)
        self.norm1 = nn.InstanceNorm1d(num_features=d_model, affine=True)
        self.activation1 = nn.GELU()

        self.conv2 = nn.Conv1d(in_channels=d_model,
                               out_channels=d_model,
                               kernel_size=3,
                               padding=1,
                               groups=d_model)
        self.norm2 = nn.InstanceNorm1d(num_features=d_model, affine=True)
        self.activation2 = nn.GELU()

    def forward(self, x):
        x = x.permute(0, 2, 1) 
        res = self.conv0(x)

        y = self.conv1(x)
        y = self.norm1(y)
        y = self.activation1(y)

        y = self.conv2(y)
        y = self.norm2(y)
        y = self.activation2(y)

        out = res + y
        return out.permute(0, 2, 1)


class ConvEmbedding(DataEmbedding):
    def __init__(self, c_in, d_model, embed_type='fixed', freq='h', dropout=0.1):
        super().__init__(c_in, d_model, embed_type, freq, dropout)
        
        # replace TokenEmbedding with ConvStem
        self.value_embedding = ConvStem(c_in, d_model)

    def forward(self, x, x_mark):
        return super().forward(x, x_mark)
  \end{lstlisting}
\end{center}

% % ------------------------------------------------------------------------------------------------------
% % -----------------------------------------------FAVOR+-------------------------------------------------
% % ------------------------------------------------------------------------------------------------------

{\noindent\hspace{-12.5pt}\normalsize\bfseries FAVOR+ wrapper}\vspace{-10pt}
\begin{center}
  \begin{lstlisting}[language=Python,
  caption={Реализация механизма линейного внимания FAVOR+ на основе библиотеки fast\_transformers.},  
  label={lst:favor}]
class FAVORAttention(nn.Module):
    def __init__(self, mask_flag, d_model, n_heads, n_random_features=256, 
                 attention_dropout=0.1, output_attention=False):
        super(FAVORAttention, self).__init__()

        self.d_k = d_model // n_heads
        self.mask_flag = mask_flag
        self.dropout = nn.Dropout(attention_dropout)

        fm_factory = Favor.factory(n_dims=n_random_features)
        if mask_flag:
            self.core = CausalLinearAttention(self.d_k, feature_map=fm_factory)
        else:
            self.core = LinearAttention(self.d_k, feature_map=fm_factory)

    def forward(self, queries, keys, values, attn_mask=None):
        B, L_Q, _, _ = queries.shape
        _, L_K, _, _ = keys.shape
        device = queries.device

        lengths = torch.full((B,), L_K, dtype=torch.int64, device=device)
        len_mask = LengthMask(lengths, max_len=L_K)

        if self.mask_flag:
            attn_mask = TriangularCausalMask(L_Q, device=device)
            lengths = torch.full((B,), L_Q, dtype=torch.int64, device=device)
            len_mask = LengthMask(lengths, max_len=L_Q)
        else:
            attn_mask = len_mask

        context = self.core(queries, keys, values,
                            attn_mask,
                            len_mask,
                            len_mask)
        
        # context = self.dropout(context)
        return context.contiguous(), None
  \end{lstlisting}
\end{center}

% {\noindent\hspace{-12.5pt}\normalsize\bfseries Import libraries}\vspace{-10pt}
% \begin{center}
%   \begin{lstlisting}[language=Python]
% import pandas as pd

% import numpy as np
% import statsmodels as sm

% import matplotlib.pyplot as plt
% import matplotlib.dates as mdates
% import matplotlib.font_manager as fm
% from matplotlib.colors import LinearSegmentedColormap

% import seaborn as sns
%   \end{lstlisting}
% \end{center}

% {\noindent\hspace{-12.5pt}\normalsize\bfseries Set constants}\vspace{-10pt}
% \begin{center}
%   \begin{lstlisting}[language=Python]
% # paths
% FONT_PATH     = '../extra/Cinzel-VariableFont_wght.ttf'
% DATASETS_PATH = '../data'
% IMAGES_PATH   = '../images'
% # colors
% RED        = '#6F1D1B'
% RICH_BLACK = '#011627'
% # font size
% SIZE_TICKS = 12
%   \end{lstlisting}
% \end{center}

% {\noindent\hspace{-12.5pt}\normalsize\bfseries Load fonts}\vspace{-10pt}
% \begin{center}
%   \begin{lstlisting}[language=Python]
% cinzel_font = fm.FontProperties(fname=FONT_PATH)
% fm.fontManager.addfont(FONT_PATH)
%   \end{lstlisting}
% \end{center}

% {\noindent\hspace{-12.5pt}\normalsize\bfseries Define styles}\vspace{-10pt}
% \begin{center}
%   \begin{lstlisting}[language=Python]
% # regular time series plot style
% classic_style = {
%     "font.family": cinzel_font.get_name(), # apply Cinzel font
%     "font.size": 16
% }

% # lag plot style
% lag_plot_style = {
%     "text.usetex": True,
%     "font.family": "serif",
%     "font.serif": ["Computer Modern Roman"],
%     "axes.grid": True,
%     "grid.color": "#8D99AE",
% }

% # acf plot style
% acf_plot_style = {
%     "font.family": cinzel_font.get_name(), # apply Cinzel font
%     "font.size": 24
% }
%   \end{lstlisting}
% \end{center}

% % -------------------------------------------------------------------------------------------------------
% % --------------------------------------REGULAR TIME SERIES PLOTS----------------------------------------
% % -------------------------------------------------------------------------------------------------------

% \begin{center}
%   \noindent\normalsize\bfseries
%   Regular Time Series Plots
% \end{center}\vspace{-17.5pt}

% \begin{center}
%   \begin{lstlisting}[language=Python]
% plt.rcdefaults() # reset to defauls
%   \end{lstlisting}
% \end{center}

% {\noindent\hspace{-12.5pt}\normalsize\bfseries Helper functions}\vspace{-10pt}
% \begin{center}
%   \begin{lstlisting}[language=Python]
% # helper function to decorate plots
% def decorate_regular_plot(ax, xname: str, 
%                               yname: str, 
%                               loc=None) -> None:
%     SIZE_TICKS = 12

%     # eliminate upper and right axes
%     ax.spines['right'].set_color('none')
%     ax.spines['top'].set_color('none')

%     # show ticks in the left and lower axes only
%     ax.xaxis.set_ticks_position('bottom')
%     ax.yaxis.set_ticks_position('left')

%     # x axis name
%     ax.set_xlabel(xname, fontsize=15)

%     # y axis name
%     ax.set_ylabel(yname, fontsize=15)

%     # adjust the font size of the tick labels
%     ax.tick_params(axis='both', which='major', labelsize=SIZE_TICKS)

%     if loc:
%         plt.legend(fontsize=10, loc=loc)

%     # adjust layout
%     plt.tight_layout()
%   \end{lstlisting}
% \end{center}

% % -----------------------------------France Electricity Consumption-------------------------------------

% {\noindent\hspace{-12.5pt}\normalsize\bfseries France Electricity Consumption}\vspace{-10pt}
% \begin{center}
%   \begin{lstlisting}[language=Python, 
%   caption={Ежемесячное потребление электричества во Франции.}, 
%   label={lst:time_series_example_France}]
% # load data
% electricity_df = pd.read_csv(
%     f'{DATASETS_PATH}/global_electricity_production_data.csv',
%     parse_dates=['date'],
%     dayfirst=False
% )

% # process data
% France_df = electricity_df[
%     (electricity_df['country_name'] == 'France') & 
%     (electricity_df['product'] == 'Electricity') & 
%     (electricity_df['parameter'] == 'Final Consumption (Calculated)')
% ].copy().set_index('date').sort_index()

% # create plot 
% with plt.rc_context(classic_style): # use context for styles not to interfere
%     fig, ax = plt.subplots(figsize=(15, 6))
%     ax.plot(France_df.index, France_df['value'], color=RED, linewidth=1.5)

%     ax.set_title('France Electricity Consumption')
%     decorate_regular_plot(ax, 'Month', 'Value (GWh)')

%     ax.xaxis.set_major_locator(mdates.YearLocator())
%     ax.xaxis.set_major_formatter(mdates.DateFormatter('%Y'))

%     plt.savefig(f'{IMAGES_PATH}/time_series_example_France.png', 
%                 dpi=300, transparent=True)

%     plt.show()
%   \end{lstlisting}
% \end{center}

% \begin{figure}[h!]
%   \centering
%   \includegraphics[width=1\textwidth, height=1\textheight, keepaspectratio]{time_series_example_France}
% \end{figure}

% % --------------------------------------------Avocado Sales----------------------------------------------

% {\noindent\hspace{-12.5pt}\normalsize\bfseries Avocado Sales}\vspace{-10pt}
% \begin{center}
%   \begin{lstlisting}[language=Python, 
%   caption={Еженедельный объём продаж авокадо}, 
%   label={lst:time_series_example_avocado}]
% # load data
% avocado_df = pd.read_csv(
%     f'{DATASETS_PATH}/avocado.csv',
%     parse_dates=['Date'],
%     dayfirst=False
% )

% # process data
% avocado_df = avocado_df.groupby('Date')['Total Volume'].mean().reset_index()
% avocado_df = avocado_df.set_index('Date').sort_index()

% # create plot
% with plt.rc_context(classic_style): # use context for styles not to interfere
%     fig, ax = plt.subplots(figsize=(15, 6))
%     ax.plot(avocado_df.index, avocado_df['Total Volume'], 
%             color=RED, linewidth=1.5)

%     ax.set_title('Avocado Sales')
%     decorate_regular_plot(ax, 'Year-Month', 'Total Number Of Avocados Sold')

%     plt.savefig(f'{IMAGES_PATH}/time_series_example_avocado.png', 
%                 dpi=300, transparent=True)

%     plt.show()
%   \end{lstlisting}
% \end{center}

% \begin{figure}[h!]
%   \centering
%   \includegraphics[width=1\textwidth, height=1\textheight, keepaspectratio]{time_series_example_avocado}
% \end{figure}

% % -------------------------------------------Wine Australia---------------------------------------------

% {\noindent\hspace{-12.5pt}\normalsize\bfseries Wine Australia}\vspace{-10pt}
% \begin{center}
%   \begin{lstlisting}[language=Python, 
%   caption={Месячный объём продаж красного вина в Австралии.}, 
%   label={lst:time_series_example_wine}]
% # load data
% wine_df = pd.read_csv(
%     f'{DATASETS_PATH}/AusWineSales.csv',
%     parse_dates=['YearMonth'],
%     dayfirst=False
% ).set_index('YearMonth').sort_index()

% # create plot
% with plt.rc_context(classic_style): # use context for styles not to interfere
%     fig, ax = plt.subplots(figsize=(15, 6))
%     ax.plot(wine_df.index, wine_df['Red'], color=RED, linewidth=1.5)

%     ax.set_title('Australian Wine Sales')
%     decorate_regular_plot(ax, 'Year', 'Wine Sales (Volume)')

%     ax.xaxis.set_major_locator(mdates.YearLocator())
%     ax.xaxis.set_major_formatter(mdates.DateFormatter('%Y'))

%     plt.savefig(f'{IMAGES_PATH}/time_series_example_wine.png', 
%                 dpi=300, transparent=True)

%     plt.show()
%   \end{lstlisting}
% \end{center}

% \begin{figure}[h!]
%   \centering
%   \includegraphics[width=1\textwidth, height=1\textheight, keepaspectratio]{time_series_example_wine}
% \end{figure}\newpage

% % -------------------------------Wine Australi Histogram---------------------------------

% {\noindent\hspace{-12.5pt}\normalsize\bfseries Wine Australia Histogram}\vspace{-10pt}
% \begin{center}
%   \begin{lstlisting}[language=Python, 
%   caption={Гистограмма месячного объёма продаж красного вина в Австралии.}, 
%   label={lst:time_series_example_wine_hist}]
% # create plot
% with plt.rc_context(classic_style): # use context for styles not to interfere
%     fig, ax = plt.subplots(figsize=(8, 6))

%     ax.hist(wine_df['Red'], color=RED, linewidth=1.5)

%     ax.set_title('Australian Wine Sales')
%     decorate_regular_plot(ax, '', '')

%     plt.grid(linewidth=0.8, linestyle='--', color='black')
%     plt.savefig(f'{IMAGES_PATH}/time_series_example_wine_hist.png', 
%                 dpi=300, transparent=True)

%     plt.show()
%   \end{lstlisting}
% \end{center}

% \begin{figure}[h!]
%   \centering
%   \includegraphics[width=0.5\textwidth, height=0.5\textheight, keepaspectratio]{time_series_example_wine_hist}
% \end{figure}\newpage

% % --------------------------------Wine Australia Density----------------------------------

% {\noindent\hspace{-12.5pt}\normalsize\bfseries Wine Australia Density}\vspace{-10pt}
% \begin{center}
%   \begin{lstlisting}[language=Python, 
%   caption={График плотности месячного объёма продаж красного вина в Австралии.}, 
%   label={lst:time_series_example_wine_density}]
% # create plot
% with plt.rc_context(classic_style): # use context for styles not to interfere
%     fig, ax = plt.subplots(figsize=(8, 6))

%     wine_df['Red'].plot(kind='kde', color=RED)

%     ax.set_title('Australian Wine Sales')
%     decorate_regular_plot(ax, '', 'Density')

%     plt.savefig(f'{IMAGES_PATH}/time_series_example_wine_density.png', 
%                 dpi=300, transparent=True)

%     plt.show()
%   \end{lstlisting}
% \end{center}

% \begin{figure}[h!]
%   \centering
%   \includegraphics[width=0.5\textwidth, height=0.5\textheight, keepaspectratio]{time_series_example_wine_density}
% \end{figure}\newpage

% % --------------------------------Wine Australia Boxplot----------------------------------

% {\noindent\hspace{-12.5pt}\normalsize\bfseries Wine Australia Boxplot}\vspace{-10pt}
% \begin{center}
%   \begin{lstlisting}[language=Python, 
%   caption={Распределения продаж красного вина в Австралии по месяцам в период с 1980 по 1996 года.}, 
%   label={lst:time_series_example_wine_boxplot}]
% wine_df['Month'] = wine_df.index.month
% wine_df['Year'] = wine_df.index.year

% pivot_df = wine_df.pivot_table(index='Year', columns='Month', values='Red')

% with plt.rc_context(classic_style): # use context for styles not to interfere
%     fig, ax = plt.subplots(figsize=(15, 6))

%     # Create box plots for each month
%     pivot_df.boxplot(
%         ax=ax,
%         boxprops=dict(color=RICH_BLACK),
%         whiskerprops=dict(color=RICH_BLACK, linestyle='--'),
%         medianprops=dict(color=RED)
%     )
%     plt.xlabel('Month')
%     plt.ylabel('Wine Sales (Volume)')
%     plt.title('Monthly Box Plot')

%     # plt.savefig(f'{IMAGES_PATH}/time_series_example_wine_boxplot.png', 
%                   dpi=300, transparent=True)

%     plt.show()
%   \end{lstlisting}
% \end{center}

% \begin{figure}[h!]
%   \centering
%   \includegraphics[width=1\textwidth, height=1\textheight, keepaspectratio]{time_series_example_wine_boxplot}
% \end{figure}\newpage

% % -------------------------------American Electric Power---------------------------------

% {\noindent\hspace{-12.5pt}\normalsize\bfseries American Electric Power}\vspace{-10pt}
% \begin{center}
%   \begin{lstlisting}[language=Python, 
%   caption={Ежечасовое потребление электричества в Америке.}, 
%   label={lst:time_series_example_aep}]
% # load data
% aep_df = pd.read_csv(
%     f'{DATASETS_PATH}/AEP_hourly.csv',
%     parse_dates=['Datetime'],
%     dayfirst=False
% ).set_index('Datetime').sort_index()

% # process data
% aep_df = aep_df.loc[f'2009']

% # create plot
% with plt.rc_context(classic_style): # use context for styles not to interfere
%     fig, ax = plt.subplots(figsize=(7, 6))
%     ax.plot(aep_df.index, aep_df['AEP_MW'], color=RED, linewidth=1.5)

%     ax.set_title('Hourly American Electricity Consumption')
%     decorate_regular_plot(ax, 'Year-Month', 'AEP Consumption (MW)')

%     plt.savefig(f'{IMAGES_PATH}/time_series_example_aep.png', 
%     dpi=300, transparent=True)

%     plt.show()
%   \end{lstlisting}
% \end{center}

% \begin{figure}[h!]
%   \centering
%   \includegraphics[width=0.5\textwidth, height=0.5\textheight, keepaspectratio]{time_series_example_aep}
% \end{figure}\newpage

% % -------------------------------American Electric Power Grouped---------------------------------

% {\noindent\hspace{-12.5pt}\normalsize\bfseries American Electric Power Grouped}\vspace{-10pt}
% \begin{center}
%   \begin{lstlisting}[language=Python, 
%   caption={Ежечасовое потребление электричества в Америке в каждом месяце.}, 
%   label={lst:time_series_example_aep_grouped}]
%   # group data by months
%   groups = aep_df.groupby(pd.Grouper(freq='ME'))
  
%   fig, ax = plt.subplots(12, 1, figsize=(10, 10))
%   ax = ax.flatten()
  
%   colors = list(mcolors.TABLEAU_COLORS.values())
  
%   # extract the values from the first 12 months
%   for i, (month_date, month_data) in enumerate(groups):
  
%       if i >= 12:
%           break
  
%       ax[i].plot(month_data.index, month_data['AEP_MW'], 
%                                    label=month_date.strftime('%B'),
%                                    color=colors[i % 10])
%       ax[i].legend(loc='upper left')
  
%       ax[i].xaxis.set_major_formatter(mdates.DateFormatter('%d')) 
  
%   # set y label
%   fig.text(x=0.04, y=0.5, s='AEP consumption (MW)', 
%            rotation=90, va='center', ha='center', 
%            fontsize=12  
%   )
  
%   # set x label
%   fig.text(x=0.5, y=0.06, s='Day of month', 
%            va='center', ha='center', fontsize=12  
%   )
  
%   plt.savefig(f'{IMAGES_PATH}/time_series_example_aep_grouped.png', 
%               dpi=300, transparent=False)
  
%   plt.show()
%   \end{lstlisting}
% \end{center}

% \begin{figure}[h!]
%   \centering
%   \includegraphics[width=0.7\textwidth, height=0.7\textheight, keepaspectratio]{time_series_example_aep_grouped}
% \end{figure}\newpage

% % ---------------------------------------------Gold Price-----------------------------------------------

% {\noindent\hspace{-12.5pt}\normalsize\bfseries Gold Price}\vspace{-10pt}
% \begin{center}
%   \begin{lstlisting}[language=Python, 
%   caption={Стоимость золота в долларах.}, 
%   label={lst:time_series_example_gold_small}]
% # load data
% gold_df = pd.read_csv(
%     f'{DATASETS_PATH}/gold_prices_quarterly.csv',
%     parse_dates=['Date'],
%     dayfirst=False
% ).set_index('Date').sort_index()

% # process data
% gold_df['USD'] = gold_df['USD'].str.replace(',', '').astype(float)

% # create plot
% with plt.rc_context(classic_style): # use context for styles not to interfere
%     fig, ax = plt.subplots(figsize=(8, 6))
%     ax.plot(gold_df.index, gold_df['USD'], color=RED, linewidth=1.5)

%     ax.set_title('Gold Price')
%     decorate_regular_plot(ax, 'Year', 'Gold Price (USD)')

%     plt.savefig(f'{IMAGES_PATH}/time_series_example_gold_small.png', 
%                 dpi=300, transparent=True)

%     plt.show()
%   \end{lstlisting}
% \end{center}

% \begin{figure}[h!]
%   \centering
%   \includegraphics[width=0.5\textwidth, height=0.5\textheight, keepaspectratio]{time_series_example_gold_small}
% \end{figure}\newpage

% % ---------------------------------------------Dow Jones Index-----------------------------------------------

% {\noindent\hspace{-12.5pt}\normalsize\bfseries Dow Jones Index}\vspace{-10pt}
% \begin{center}
%   \begin{lstlisting}[language=Python, 
%   caption={Промышленный индекс Доу Джонса.}, 
%   label={lst:time_series_example_Dow_Jones_small}]
% # load data
% dowJones_df = pd.read_csv(
%     f'{DATASETS_PATH}/DJIA.csv',
%     parse_dates=['observation_date'],
%     dayfirst=False
% ).set_index('observation_date').sort_index()

% # create plot
% with plt.rc_context(classic_style): # use context for styles not to interfere
%     fig, ax = plt.subplots(figsize=(8, 6))
%     ax.plot(dowJones_df.index, dowJones_df['DJIA'], color=RED, linewidth=1.5)

%     ax.set_title('Dow Jones Industrial Average')
%     decorate_regular_plot(ax, 'Year-Month', 'Index')

%     plt.savefig(f'{IMAGES_PATH}/time_series_example_Dow_Jones_small.png', 
%                 dpi=300, transparent=True)

%     plt.show()
%   \end{lstlisting}
% \end{center}

% \begin{figure}[h!]
%   \centering
%   \includegraphics[width=0.5\textwidth, height=0.5\textheight, keepaspectratio]{time_series_example_Dow_Jones_small}
% \end{figure}\newpage

% % --------------------------------------Change In Dow Jones Index----------------------------------------

% \begin{center}
%   \begin{lstlisting}[language=Python, 
%   caption={Изменение в индексе Доу Джонса.}, 
%   label={lst:time_series_example_Dow_Jones_change_small}]
% # process data
% dowJones_df['change'] = dowJones_df['DJIA'].diff()

% # create plot
% with plt.rc_context(classic_style): # use context for styles not to interfere
%     fig, ax = plt.subplots(figsize=(8, 6))
%     ax.plot(dowJones_df.index, dowJones_df['change'], color=RED, linewidth=1.5)

%     ax.set_title('Change in Dow Jones Index')
%     decorate_regular_plot(ax, 'Year-Month', 'Index')

%     # plt.savefig(f'{IMAGES_PATH}/time_series_example_Dow_Jones_change_small.png', dpi=300, transparent=True)

%     plt.show()
%   \end{lstlisting}
% \end{center}

% \begin{figure}[h!]
%   \centering
%   \includegraphics[width=0.5\textwidth, height=0.5\textheight, keepaspectratio]{time_series_example_Dow_Jones_change_small}
% \end{figure}\newpage

% % -----------------------------------------------Random-------------------------------------------------

% \begin{center}
%   \begin{lstlisting}[language=Python, 
%   caption={Белый шум.}, 
%   label={lst:time_series_example_random_small}]
% # generate a normal distribution sample
% sample_size=1000
% random_series = pd.DataFrame(np.random.normal(size=sample_size)) 

% # create plot
% with plt.rc_context(classic_style): # use context for styles not to interfere
%     fig, ax = plt.subplots(figsize=(8, 6))
%     ax.plot(random_series, color=RED, linewidth=1.5)

%     ax.set_title('White Noise')
%     decorate_regular_plot(ax, '', '')

%     plt.savefig(f'{IMAGES_PATH}/time_series_example_random_small.png', 
%                 dpi=300, transparent=True)

%     plt.show()
%   \end{lstlisting}
% \end{center}

% \begin{figure}[h!]
%   \centering
%   \includegraphics[width=0.5\textwidth, height=0.5\textheight, keepaspectratio]{time_series_example_random_small}
% \end{figure}\newpage

% % -----------------------------------------------Sunspot Number-------------------------------------------------

% \begin{center}
%   \begin{lstlisting}[language=Python, 
%   caption={Среднегодовое общее количество солнечных пятен \cite{silso}.}, 
%   label={lst:time_series_example_sunspots_small}]
% # load data
% sunspot_df = pd.read_csv(
%     f'{DATASETS_PATH}/sunspots.csv',
%     sep=';', 
%     header=None, 
%     names=['Year', 'Sunspots', 'Col3', 'Col4', 'Col5']
% ).set_index('Year').sort_index()

% # process data
% sunspot_df = sunspot_df[sunspot_df.index <= 1880.5]

% # create plot
% with plt.rc_context(classic_style): # use context for styles not to interfere
%     fig, ax = plt.subplots(figsize=(8, 6))
%     ax.plot(sunspot_df.index, sunspot_df['Sunspots'], color=RED, linewidth=1.5)

%     ax.set_title('Yearly Mean Total Sunspot Number')
%     decorate_regular_plot(ax, 'Year', 'Sunspot number')

%     plt.savefig(f'{IMAGES_PATH}/time_series_example_sunspots_small.png', 
%                 dpi=300, transparent=True)

%     plt.show()
%   \end{lstlisting}
% \end{center}

% \begin{figure}[h!]
%   \centering
%   \includegraphics[width=0.5\textwidth, height=0.5\textheight, keepaspectratio]{time_series_example_sunspots_small}
% \end{figure}\newpage

% % -----------------------------------------Airline Passengers-------------------------------------------

% \begin{center}
%   \begin{lstlisting}[language=Python, 
%   caption={Пассажиры авиалиний.}, 
%   label={lst:time_series_example_airline_small}]
% # load data
% airline_df = pd.read_csv(
%     f'{DATASETS_PATH}/airline-passengers.csv',
%     parse_dates=['Month'],
%     dayfirst=False
% ).set_index('Month').sort_index()

% # create plot
% with plt.rc_context(classic_style): # use context for styles not to interfere
%     fig, ax = plt.subplots(figsize=(8, 6))
%     ax.plot(airline_df.index, airline_df['Passengers'], color=RED, linewidth=1.5)

%     ax.set_title('Monthly Airline Passengers (1949–1960)')
%     decorate_regular_plot(ax, 'Year', 'Number of Passengers')

%     plt.savefig(f'{IMAGES_PATH}/time_series_example_airline_small.png', 
%                 dpi=300, transparent=True)

%     plt.show()
%   \end{lstlisting}
% \end{center}

% \begin{figure}[h!]
%   \centering
%   \includegraphics[width=0.5\textwidth, height=0.5\textheight, keepaspectratio]{time_series_example_airline_small}
% \end{figure}\newpage

% % -----------------------------Airline Passengers Box-Cox---------------------------------

% \begin{center}
%   \begin{lstlisting}[language=Python, 
%   caption={Пассажиры авиалиний (после преобразования Бокса-Кокса).}, 
%   label={lst:time_series_example_airline_boxcox_small}]
% # apply box cox transformation
% airline_df_stabilize, lambda_value = sp.stats.boxcox(airline_df['Passengers'])

% # create plot
% with plt.rc_context(classic_style): # use context for styles not to interfere
%     fig, ax = plt.subplots(figsize=(8, 6))
%     ax.plot(airline_df.index, airline_df_stabilize, color=RED, linewidth=1.5)

%     ax.set_title('Stabilized Monthly Airline Passengers')
%     decorate_regular_plot(ax, 'Year', 'Number of Passengers')

%     plt.savefig(f'{IMAGES_PATH}/time_series_example_airline_boxcox_small.png', 
%                 dpi=300, transparent=True)

%     plt.show()
%   \end{lstlisting}
% \end{center}

% \begin{figure}[h!]
%   \centering
%   \includegraphics[width=0.5\textwidth, height=0.5\textheight, keepaspectratio]{time_series_example_airline_boxcox_small}
% \end{figure}\newpage

% % -------------------------------------------------------------------------------------------------------
% % ----------------------------------------------LAG PLOTS------------------------------------------------
% % -------------------------------------------------------------------------------------------------------

% \begin{center}
%   \noindent\normalsize\bfseries
%   Lag Plots
% \end{center}\vspace{-17.5pt}

% \begin{center}
%   \begin{lstlisting}[language=Python]
% plt.rcdefaults() # reset to defauls
%   \end{lstlisting}
% \end{center}

% {\noindent\hspace{-12.5pt}\normalsize\bfseries Helper functions}\vspace{-10pt}
% \begin{center}
%   \begin{lstlisting}[language=Python]
% # helper function to decorate plots
% def decorate_lag_plot(ax, xname: str, 
%                           yname: str, 
%                           loc=None) -> None:
%     SIZE_TICKS = 12

%     # x and y axis labels
%     ax.set_xlabel(xname, fontsize=20)
%     ax.set_ylabel(yname, fontsize=20)

%     # adjust tick labels size
%     ax.tick_params(axis='both', which='major', labelsize=SIZE_TICKS)

%     if loc:
%         ax.legend(fontsize=10, loc=loc)

%     plt.grid(True, linestyle='--', linewidth=0.05)
%     plt.tight_layout()

% # main function for creating lag plots
% def lag_plot(df: pd.DataFrame, 
%              value_name: str, 
%              l: int=1, 
%              file_name: str=None) -> None:
%     lag_df = pd.DataFrame(df[value_name])
%     lag_df['lag'] = df[value_name].shift(l)

%     lag_df.dropna(inplace=True)

%     _, ax = plt.subplots(figsize=(7, 6))

%     decorate_lag_plot(ax, r'$y_t$', f'$y_{{t + {l}}}$', '')

%     sns.scatterplot(data=lag_df, 
%                     x=value_name,
%                     y="lag", 
%                     color=RICH_BLACK, 
%                     size=100, 
%                     legend=False, 
%                     ax=ax)

%     if file_name:
%         plt.savefig(f'{IMAGES_PATH}/{file_name}.png', 
%                     dpi=300, transparent=True)

%     plt.show()
%   \end{lstlisting}
% \end{center}

% % -------------------------------------------Wine Australia---------------------------------------------

% {\noindent\hspace{-12.5pt}\normalsize\bfseries Wine Australia}\vspace{-10pt}
% \begin{center}
%   \begin{lstlisting}[language=Python, 
%   caption={График задержек продаж вина в Австралии.}, 
%   label={lst:time_series_lag_plot_wine}]
% # load data
% wine_df = pd.read_csv(
%     f'{DATASETS_PATH}/AusWineSales.csv',
%     parse_dates=['YearMonth'],
%     dayfirst=False
% ).set_index('YearMonth').sort_index()

% # create plot
% with plt.rc_context(lag_plot_style): # use context for styles not to interfere
%     lag_plot(wine_df, 'Red', l=1, file_name=None)
%   \end{lstlisting}
% \end{center}

% \begin{figure}[h!]
%   \centering
%   \includegraphics[width=0.6\textwidth, height=0.6\textheight, keepaspectratio]{australia_wine_lag_plot}
% \end{figure}\newpage

% % -----------------------------------------------Random-------------------------------------------------

% {\noindent\hspace{-12.5pt}\normalsize\bfseries Random}\vspace{-10pt}
% \begin{center}
%   \begin{lstlisting}[language=Python, 
%   caption={График задержек белого шума.}, 
%   label={lst:time_series_lag_plot_random}]
% # generate a normal distribution sample
% sample_size=1000
% random_series = pd.DataFrame(np.random.normal(size=sample_size)) 

% # create plot
% with plt.rc_context(lag_plot_style): # use context for styles not to interfere
%     lag_plot(random_series, 0, l=1, file_name=None)
%   \end{lstlisting}
% \end{center}

% \begin{figure}[h!]
%   \centering
%   \includegraphics[width=0.6\textwidth, height=0.6\textheight, keepaspectratio]{random_series_lag_plot}
% \end{figure}\newpage

% % ------------------------------------------------------------------------------------------------------
% % ----------------------------------------CORRELATION MATRICES------------------------------------------
% % ------------------------------------------------------------------------------------------------------

% \begin{center}
%   \noindent\normalsize\bfseries
%   Correlation Matrices
% \end{center}\vspace{-17.5pt}

% \begin{center}
%   \begin{lstlisting}[language=Python]
% plt.rcdefaults() # reset to defauls
%   \end{lstlisting}
% \end{center}

% {\noindent\hspace{-12.5pt}\normalsize\bfseries Helper functions}\vspace{-10pt}
% \begin{center}
%   \begin{lstlisting}[language=Python]
% # main function for creating correlation matrix plots
% def corr_matrix_plot(df: pd.DataFrame, value_name: str, 
%                                        n: int=10, 
%                                        file_name: str=None) -> None:
%     lag_df = pd.DataFrame(df[value_name]).rename(columns={value_name: '$y_t$'})
%     for l in range(1, n+1):
%         lag_df[f'$y_{{t + {l}}}$'] = df[value_name].shift(l)

%     lag_df.dropna(inplace=True)

%     # compute the correlation matrix
%     corr = lag_df.corr()

%     # generate a mask for the upper triangle
%     mask = np.triu(np.ones_like(corr, dtype=bool))

%     # set up the matplotlib figure
%     _, _ = plt.subplots(figsize=(11, 9))

%     # generate a custom diverging colormap
%     cmap = LinearSegmentedColormap.from_list('custom_cmap', [RICH_BLACK, RED])

%     # draw the heatmap with the mask and correct aspect ratio
%     sns.heatmap(corr, mask=mask, cmap=cmap, vmax=.3, center=0,
%                       square=True, linewidths=.5, cbar_kws={"shrink": .5})

%     if file_name:
%         plt.savefig(f'{IMAGES_PATH}/{file_name}.png', 
%                     dpi=300, transparent=True)

%     plt.show()
%   \end{lstlisting}
% \end{center}

% % --------------------------------------------Avocado Sales----------------------------------------------

% {\noindent\hspace{-12.5pt}\normalsize\bfseries Avocado Sales}\vspace{-10pt}
% \begin{center}
%   \begin{lstlisting}[language=Python, 
%   caption={Диагональная корреляционная матрица объема продаж авокадо.}, 
%   label={lst:correlation_matrix_avocado}]
% # load data
% avocado_df = pd.read_csv(
%     f'{DATASETS_PATH}/avocado.csv',
%     parse_dates=['Date'],
%     dayfirst=False
% ).set_index('Date').sort_index()

% # process data
% avocado_df = avocado_df.groupby('Date')['Total Volume'].mean().reset_index()
% avocado_df = avocado_df.set_index('Date').sort_index()

% # create plot
% corr_matrix_plot(avocado_df, 'Total Volume', n=20, file_name=None)
%   \end{lstlisting}
% \end{center}

% \begin{figure}[h!]
%   \centering
%   \includegraphics[width=0.7\textwidth, height=0.7\textheight, keepaspectratio]{correlation_matrix_avocado}
% \end{figure}

% % ------------------------------------------------------------------------------------------------------
% % -----------------------------------AUTOCORRELATION FUNCTION (ACF)-------------------------------------
% % ------------------------------------------------------------------------------------------------------

% \begin{center}
%   \noindent\normalsize\bfseries
%   Autocorrelation Function (ACF)
% \end{center}\vspace{-17.5pt}

% \begin{center}
%   \begin{lstlisting}[language=Python]
% plt.rcdefaults() # reset to defauls
%   \end{lstlisting}
% \end{center}

% {\noindent\hspace{-12.5pt}\normalsize\bfseries Helper functions}\vspace{-10pt}
% \begin{center}
%   \begin{lstlisting}[language=Python]
% # helper function to decorate plots
% def decorate_acf(ax, loc=None) -> None:
%     SIZE_TICKS = 24

%     # Eliminate upper and right axes
%     ax.spines['right'].set_color('none')
%     ax.spines['top'].set_color('none')

%     # Show ticks in the left and lower axes only
%     ax.xaxis.set_ticks_position('bottom')
%     ax.yaxis.set_ticks_position('left')

%     # Adjust the font size of the tick labels
%     ax.tick_params(axis='both', which='major', labelsize=SIZE_TICKS)

%     if loc:
%         plt.legend(fontsize=10, loc=loc)

%     # Adjust layout
%     plt.tight_layout()
%   \end{lstlisting}
% \end{center}

% % -------------------------------------------Wine Australia---------------------------------------------

% {\noindent\hspace{-12.5pt}\normalsize\bfseries Wine Australia}\vspace{-10pt}
% \begin{center}
%   \begin{lstlisting}[language=Python, 
%   caption={Коррелограмма для объёма продаж красного вина в Австралии.}, 
%   label={lst:acf_wine}]
% # load data
% wine_df = pd.read_csv(
%     f'{DATASETS_PATH}/AusWineSales.csv',
%     parse_dates=['YearMonth'],
%     dayfirst=False
% ).set_index('YearMonth').sort_index()

% # create plot
% with plt.rc_context(acf_plot_style): # use context for styles not to interfere
%     _, ax = plt.subplots(figsize=(20, 6))

%     decorate_acf(ax)

%     sm.graphics.tsaplots.plot_acf(wine_df['Red'], 
%                                   lags=np.arange(0, 100, 1), 
%                                   ax=ax, 
%                                   color=RICH_BLACK, 
%                                   vlines_kwargs={'colors': RICH_BLACK, 
%                                                  'linewidth': 1.5});

%     ax.set_ylim(-0.75, 1)
%     ax.grid(False)

%     plt.savefig(f'{IMAGES_PATH}/acf_wine.png', 
%                 dpi=300, transparent=True)

%     plt.show();
%   \end{lstlisting}
% \end{center}

% \begin{figure}[h!]
%   \centering
%   \includegraphics[width=1\textwidth, height=1\textheight, keepaspectratio]{acf_wine}
% \end{figure}

% -----------------------------------------------CONTENT-------------------------------------------------

% remove centering from sections
\titleformat{\section}
  {\normalfont\large\bfseries}{\thesection}{1em}{}
